% Options for packages loaded elsewhere
% Options for packages loaded elsewhere
\PassOptionsToPackage{unicode}{hyperref}
\PassOptionsToPackage{hyphens}{url}
%
\documentclass[
  ignorenonframetext,
  aspectratio=169,
  russian,
]{beamer}
\newif\ifbibliography
\usepackage{pgfpages}
\setbeamertemplate{caption}[numbered]
\setbeamertemplate{caption label separator}{: }
\setbeamercolor{caption name}{fg=normal text.fg}
\beamertemplatenavigationsymbolshorizontal
% Prevent slide breaks in the middle of a paragraph
\widowpenalties 1 10000
\raggedbottom
\AtBeginPart{
  \frame{\partpage}
}
\AtBeginSection{
  \ifbibliography
  \else
    \frame{\sectionpage}
  \fi
}
\AtBeginSubsection{
  \frame{\subsectionpage}
}
\usepackage{iftex}
\ifPDFTeX
  \usepackage[T1]{fontenc}
  \usepackage[utf8]{inputenc}
  \usepackage{textcomp} % provide euro and other symbols
\else % if luatex or xetex
  \usepackage{unicode-math} % this also loads fontspec
  \defaultfontfeatures{Scale=MatchLowercase}
  \defaultfontfeatures[\rmfamily]{Ligatures=TeX,Scale=1}
\fi
\usepackage{lmodern}

\usetheme[]{Montpellier}
\usecolortheme[]{seagull}
\ifPDFTeX\else
  % xetex/luatex font selection
\fi
% Use upquote if available, for straight quotes in verbatim environments
\IfFileExists{upquote.sty}{\usepackage{upquote}}{}
\IfFileExists{microtype.sty}{% use microtype if available
  \usepackage[]{microtype}
  \UseMicrotypeSet[protrusion]{basicmath} % disable protrusion for tt fonts
}{}

\usepackage{color}
\usepackage{fancyvrb}
\newcommand{\VerbBar}{|}
\newcommand{\VERB}{\Verb[commandchars=\\\{\}]}
\DefineVerbatimEnvironment{Highlighting}{Verbatim}{commandchars=\\\{\}}
% Add ',fontsize=\small' for more characters per line
\usepackage{framed}
\definecolor{shadecolor}{RGB}{241,243,245}
\newenvironment{Shaded}{\begin{snugshade}}{\end{snugshade}}
\newcommand{\AlertTok}[1]{\textcolor[rgb]{0.68,0.00,0.00}{#1}}
\newcommand{\AnnotationTok}[1]{\textcolor[rgb]{0.37,0.37,0.37}{#1}}
\newcommand{\AttributeTok}[1]{\textcolor[rgb]{0.40,0.45,0.13}{#1}}
\newcommand{\BaseNTok}[1]{\textcolor[rgb]{0.68,0.00,0.00}{#1}}
\newcommand{\BuiltInTok}[1]{\textcolor[rgb]{0.00,0.23,0.31}{#1}}
\newcommand{\CharTok}[1]{\textcolor[rgb]{0.13,0.47,0.30}{#1}}
\newcommand{\CommentTok}[1]{\textcolor[rgb]{0.37,0.37,0.37}{#1}}
\newcommand{\CommentVarTok}[1]{\textcolor[rgb]{0.37,0.37,0.37}{\textit{#1}}}
\newcommand{\ConstantTok}[1]{\textcolor[rgb]{0.56,0.35,0.01}{#1}}
\newcommand{\ControlFlowTok}[1]{\textcolor[rgb]{0.00,0.23,0.31}{\textbf{#1}}}
\newcommand{\DataTypeTok}[1]{\textcolor[rgb]{0.68,0.00,0.00}{#1}}
\newcommand{\DecValTok}[1]{\textcolor[rgb]{0.68,0.00,0.00}{#1}}
\newcommand{\DocumentationTok}[1]{\textcolor[rgb]{0.37,0.37,0.37}{\textit{#1}}}
\newcommand{\ErrorTok}[1]{\textcolor[rgb]{0.68,0.00,0.00}{#1}}
\newcommand{\ExtensionTok}[1]{\textcolor[rgb]{0.00,0.23,0.31}{#1}}
\newcommand{\FloatTok}[1]{\textcolor[rgb]{0.68,0.00,0.00}{#1}}
\newcommand{\FunctionTok}[1]{\textcolor[rgb]{0.28,0.35,0.67}{#1}}
\newcommand{\ImportTok}[1]{\textcolor[rgb]{0.00,0.46,0.62}{#1}}
\newcommand{\InformationTok}[1]{\textcolor[rgb]{0.37,0.37,0.37}{#1}}
\newcommand{\KeywordTok}[1]{\textcolor[rgb]{0.00,0.23,0.31}{\textbf{#1}}}
\newcommand{\NormalTok}[1]{\textcolor[rgb]{0.00,0.23,0.31}{#1}}
\newcommand{\OperatorTok}[1]{\textcolor[rgb]{0.37,0.37,0.37}{#1}}
\newcommand{\OtherTok}[1]{\textcolor[rgb]{0.00,0.23,0.31}{#1}}
\newcommand{\PreprocessorTok}[1]{\textcolor[rgb]{0.68,0.00,0.00}{#1}}
\newcommand{\RegionMarkerTok}[1]{\textcolor[rgb]{0.00,0.23,0.31}{#1}}
\newcommand{\SpecialCharTok}[1]{\textcolor[rgb]{0.37,0.37,0.37}{#1}}
\newcommand{\SpecialStringTok}[1]{\textcolor[rgb]{0.13,0.47,0.30}{#1}}
\newcommand{\StringTok}[1]{\textcolor[rgb]{0.13,0.47,0.30}{#1}}
\newcommand{\VariableTok}[1]{\textcolor[rgb]{0.07,0.07,0.07}{#1}}
\newcommand{\VerbatimStringTok}[1]{\textcolor[rgb]{0.13,0.47,0.30}{#1}}
\newcommand{\WarningTok}[1]{\textcolor[rgb]{0.37,0.37,0.37}{\textit{#1}}}

\usepackage{longtable,booktabs,array}
\usepackage{calc} % for calculating minipage widths
\usepackage{caption}
% Make caption package work with longtable
\makeatletter
\def\fnum@table{\tablename~\thetable}
\makeatother
\usepackage{graphicx}
\makeatletter
\newsavebox\pandoc@box
\newcommand*\pandocbounded[1]{% scales image to fit in text height/width
  \sbox\pandoc@box{#1}%
  \Gscale@div\@tempa{\textheight}{\dimexpr\ht\pandoc@box+\dp\pandoc@box\relax}%
  \Gscale@div\@tempb{\linewidth}{\wd\pandoc@box}%
  \ifdim\@tempb\p@<\@tempa\p@\let\@tempa\@tempb\fi% select the smaller of both
  \ifdim\@tempa\p@<\p@\scalebox{\@tempa}{\usebox\pandoc@box}%
  \else\usebox{\pandoc@box}%
  \fi%
}
% Set default figure placement to htbp
\def\fps@figure{htbp}
\makeatother



\ifLuaTeX
\usepackage[bidi=basic,provide=*]{babel}
\else
\usepackage[bidi=default,provide=*]{babel}
\fi
% get rid of language-specific shorthands (see #6817):
\let\LanguageShortHands\languageshorthands
\def\languageshorthands#1{}


\setlength{\emergencystretch}{3em} % prevent overfull lines

\providecommand{\tightlist}{%
  \setlength{\itemsep}{0pt}\setlength{\parskip}{0pt}}



 

\usepackage[]{csquotes}

\usepackage{libertine}
\makeatletter
\@ifpackageloaded{caption}{}{\usepackage{caption}}
\AtBeginDocument{%
\ifdefined\contentsname
  \renewcommand*\contentsname{Содержание}
\else
  \newcommand\contentsname{Содержание}
\fi
\ifdefined\listfigurename
  \renewcommand*\listfigurename{Список иллюстраций}
\else
  \newcommand\listfigurename{Список иллюстраций}
\fi
\ifdefined\listtablename
  \renewcommand*\listtablename{Список таблиц}
\else
  \newcommand\listtablename{Список таблиц}
\fi
\ifdefined\figurename
  \renewcommand*\figurename{Рисунок}
\else
  \newcommand\figurename{Рисунок}
\fi
\ifdefined\tablename
  \renewcommand*\tablename{Таблица}
\else
  \newcommand\tablename{Таблица}
\fi
}
\@ifpackageloaded{float}{}{\usepackage{float}}
\floatstyle{ruled}
\@ifundefined{c@chapter}{\newfloat{codelisting}{h}{lop}}{\newfloat{codelisting}{h}{lop}[chapter]}
\floatname{codelisting}{Список}
\newcommand*\listoflistings{\listof{codelisting}{Листинги}}
\makeatother
\makeatletter
\makeatother
\makeatletter
\@ifpackageloaded{caption}{}{\usepackage{caption}}
\@ifpackageloaded{subcaption}{}{\usepackage{subcaption}}
\makeatother

\usepackage{bookmark}
\IfFileExists{xurl.sty}{\usepackage{xurl}}{} % add URL line breaks if available
\urlstyle{same}
\hypersetup{
  pdftitle={Лабораторная работа №14},
  pdfauthor={Mohamed Musa},
  pdflang={ru-RU},
  hidelinks,
  pdfcreator={LaTeX via pandoc}}


\title{Лабораторная работа №14}
\subtitle{Реализация продвинутых механизмов в shell}
\author{Mohamed Musa}
\date{2025-10-09}

\begin{document}
\frame{\titlepage}

\renewcommand*\contentsname{Содержание}
\begin{frame}[allowframebreaks]
  \frametitle{Содержание}
  \setcounter{tocdepth}{2}
  \tableofcontents
\end{frame}
\setcounter{tocdepth}{2}
\tableofcontents
}

\section{1. Информация}\label{ux438ux43dux444ux43eux440ux43cux430ux446ux438ux44f}

\begin{frame}{1.1 Докладчик}
\phantomsection\label{ux434ux43eux43aux43bux430ux434ux447ux438ux43a}
\begin{columns}[c]
\begin{column}{0.7\linewidth}
\begin{itemize}[<+->]
\tightlist
\item
  Mohamed Musa
\item
  студент
\item
  Российский университет дружбы народов им. П. Лумумбы
\item
  1032248286@pfur.ru
\end{itemize}
\end{column}

\begin{column}{0.3\linewidth}
\pandocbounded{\includegraphics[keepaspectratio]{./image/kulyabov.jpg}}
\end{column}
\end{columns}
\end{frame}

\section{2. Вводная
часть}\label{ux432ux432ux43eux434ux43dux430ux44f-ux447ux430ux441ux442ux44c}

\begin{frame}{2.1 Актуальность}
\phantomsection\label{ux430ux43aux442ux443ux430ux43bux44cux43dux43eux441ux442ux44c}
\begin{itemize}[<+->]
\tightlist
\item
  Изучение продвинутых концепций программирования в Linux
\item
  Реализация упрощённого механизма семафоров
\item
  Создание собственной реализации команды man
\item
  Генерация случайных последовательностей с помощью \$RANDOM
\end{itemize}
\end{frame}

\begin{frame}{2.2 Объект и предмет исследования}
\phantomsection\label{ux43eux431ux44aux435ux43aux442-ux438-ux43fux440ux435ux434ux43cux435ux442-ux438ux441ux441ux43bux435ux434ux43eux432ux430ux43dux438ux44f}
\begin{itemize}[<+->]
\tightlist
\item
  Shell-скрипты и их возможности
\item
  Механизмы синхронизации процессов
\item
  Системные справочные страницы (man pages)
\item
  Генерация псевдослучайных чисел в Bash
\end{itemize}
\end{frame}

\begin{frame}{2.3 Цели и задачи}
\phantomsection\label{ux446ux435ux43bux438-ux438-ux437ux430ux434ux430ux447ux438}
\begin{itemize}[<+->]
\tightlist
\item
  Реализовать упрощённый механизм семафоров
\item
  Создать собственную реализацию команды man
\item
  Разработать генератор случайных последовательностей
\item
  Практиковать продвинутые концепции программирования в shell
\end{itemize}
\end{frame}

\section{3. Основные
понятия}\label{ux43eux441ux43dux43eux432ux43dux44bux435-ux43fux43eux43dux44fux442ux438ux44f}

\begin{frame}{3.1 Семафоры}
\phantomsection\label{ux441ux435ux43cux430ux444ux43eux440ux44b}
\begin{itemize}[<+->]
\tightlist
\item
  Семафор - переменная для синхронизации доступа к общему ресурсу
\item
  Используется для предотвращения конфликта между процессами
\item
  Двоичный семафор: 0 - ресурс занят, 1 - ресурс свободен
\item
  В shell реализуется через файл-блокировку
\end{itemize}
\end{frame}

\begin{frame}{3.2 Команда man}
\phantomsection\label{ux43aux43eux43cux430ux43dux434ux430-man}
\begin{itemize}[<+->]
\tightlist
\item
  Справочная система Linux
\item
  Структура: man1, man2, \ldots, man8 (разделы)
\item
  Формат: /usr/share/man/man{[}номер\_раздела{]}/имя.номер\_раздела.gz
\item
  Пример: /usr/share/man/man1/ls.1.gz
\end{itemize}
\end{frame}

\begin{frame}{3.3 Генерация случайных чисел}
\phantomsection\label{ux433ux435ux43dux435ux440ux430ux446ux438ux44f-ux441ux43bux443ux447ux430ux439ux43dux44bux445-ux447ux438ux441ux435ux43b}
\begin{itemize}[<+->]
\tightlist
\item
  В Bash переменная \$RANDOM генерирует числа от 0 до 32767
\item
  Используется для получения псевдослучайных значений
\item
  Для получения букв используем: \((printf '\\\)(printf \enquote*{\%03o}
  \$((97 + RANDOM \% 26)))')
\end{itemize}
\end{frame}

\section{4. Практическая
реализация}\label{ux43fux440ux430ux43aux442ux438ux447ux435ux441ux43aux430ux44f-ux440ux435ux430ux43bux438ux437ux430ux446ux438ux44f}

\begin{frame}[fragile]{4.1 Задание 1: Семафор}
\phantomsection\label{ux437ux430ux434ux430ux43dux438ux435-1-ux441ux435ux43cux430ux444ux43eux440}
\begin{Shaded}
\begin{Highlighting}[]
\CommentTok{\#!/bin/bash}
\CommentTok{\# semaphore.sh {-} упрощённый механизм семафоров}
\VariableTok{RESOURCE\_LOCK}\OperatorTok{=}\StringTok{"/tmp/semaphore.lock"}
\VariableTok{TIME\_WAIT}\OperatorTok{=}\VariableTok{$\{1}\OperatorTok{:{-}}\NormalTok{5}\VariableTok{\}} \CommentTok{\# время ожидания}
\VariableTok{TIME\_USE}\OperatorTok{=}\VariableTok{$\{2}\OperatorTok{:{-}}\NormalTok{3}\VariableTok{\}}   \CommentTok{\# время использования}
\VariableTok{PROCESS\_NAME}\OperatorTok{=}\VariableTok{$\{3}\OperatorTok{:{-}}\StringTok{"Process }\VariableTok{$$}\StringTok{"}\VariableTok{\}}

\BuiltInTok{echo} \StringTok{"}\VariableTok{$PROCESS\_NAME}\StringTok{: Попытка захвата ресурса..."}
\ControlFlowTok{while} \BuiltInTok{[} \OtherTok{{-}f} \StringTok{"}\VariableTok{$RESOURCE\_LOCK}\StringTok{"} \BuiltInTok{]}\KeywordTok{;} \ControlFlowTok{do}
    \BuiltInTok{echo} \StringTok{"}\VariableTok{$PROCESS\_NAME}\StringTok{: Ресурс занят, ожидание..."}
    \FunctionTok{sleep}\NormalTok{ 1}
\ControlFlowTok{done}

\BuiltInTok{echo} \VariableTok{$$} \OperatorTok{\textgreater{}} \StringTok{"}\VariableTok{$RESOURCE\_LOCK}\StringTok{"}
\BuiltInTok{echo} \StringTok{"}\VariableTok{$PROCESS\_NAME}\StringTok{: Ресурс захвачен, использование в течение }\VariableTok{$TIME\_USE}\StringTok{ секунд..."}
\FunctionTok{sleep} \VariableTok{$TIME\_USE}
\FunctionTok{rm} \AttributeTok{{-}f} \StringTok{"}\VariableTok{$RESOURCE\_LOCK}\StringTok{"}
\BuiltInTok{echo} \StringTok{"}\VariableTok{$PROCESS\_NAME}\StringTok{: Ресурс освобождён"}
\end{Highlighting}
\end{Shaded}
\end{frame}

\begin{frame}[fragile]{4.2 Задание 2: Команда man}
\phantomsection\label{ux437ux430ux434ux430ux43dux438ux435-2-ux43aux43eux43cux430ux43dux434ux430-man}
\begin{Shaded}
\begin{Highlighting}[]
\CommentTok{\#!/bin/bash}
\CommentTok{\# myman.sh {-} реализация команды man}
\VariableTok{COMMAND}\OperatorTok{=}\VariableTok{$1}
\ControlFlowTok{if} \BuiltInTok{[} \OtherTok{{-}z} \StringTok{"}\VariableTok{$COMMAND}\StringTok{"} \BuiltInTok{]}\KeywordTok{;} \ControlFlowTok{then}
    \BuiltInTok{echo} \StringTok{"Использование: }\VariableTok{$0}\StringTok{ \textless{}команда\textgreater{}"}
    \BuiltInTok{exit}\NormalTok{ 1}
\ControlFlowTok{fi}

\VariableTok{MANPAGE}\OperatorTok{=}\StringTok{"/usr/share/man/man1/}\VariableTok{$COMMAND}\StringTok{.1.gz"}
\ControlFlowTok{if} \BuiltInTok{[} \OtherTok{{-}f} \StringTok{"}\VariableTok{$MANPAGE}\StringTok{"} \BuiltInTok{]}\KeywordTok{;} \ControlFlowTok{then}
    \FunctionTok{zcat} \StringTok{"}\VariableTok{$MANPAGE}\StringTok{"} \KeywordTok{|} \FunctionTok{less}
\ControlFlowTok{else}
    \BuiltInTok{echo} \StringTok{"Справка для команды \textquotesingle{}}\VariableTok{$COMMAND}\StringTok{\textquotesingle{} не найдена."}
    \BuiltInTok{exit}\NormalTok{ 1}
\ControlFlowTok{fi}
\end{Highlighting}
\end{Shaded}
\end{frame}

\begin{frame}[fragile]{4.3 Задание 3: Генерация случайных
последовательностей}
\phantomsection\label{ux437ux430ux434ux430ux43dux438ux435-3-ux433ux435ux43dux435ux440ux430ux446ux438ux44f-ux441ux43bux443ux447ux430ux439ux43dux44bux445-ux43fux43eux441ux43bux435ux434ux43eux432ux430ux442ux435ux43bux44cux43dux43eux441ux442ux435ux439}
\begin{Shaded}
\begin{Highlighting}[]
\CommentTok{\#!/bin/bash}
\CommentTok{\# random\_seq.sh {-} генератор случайной последовательности}
\VariableTok{LENGTH}\OperatorTok{=}\VariableTok{$\{1}\OperatorTok{:{-}}\NormalTok{10}\VariableTok{\}}

\FunctionTok{generate\_random\_char()} \KeywordTok{\{}
    \BuiltInTok{local} \VariableTok{random\_num}\OperatorTok{=}\VariableTok{$((RANDOM} \OperatorTok{\%} \DecValTok{26}\VariableTok{))}
    \BuiltInTok{local} \VariableTok{char}\OperatorTok{=}\VariableTok{$(}\BuiltInTok{printf} \StringTok{"}\DataTypeTok{\textbackslash{}\textbackslash{}}\VariableTok{$(}\BuiltInTok{printf} \StringTok{\textquotesingle{}\%03o\textquotesingle{}} \VariableTok{$((}\DecValTok{97} \OperatorTok{+} \VariableTok{random\_num)))}\StringTok{"}\VariableTok{)}
    \BuiltInTok{echo} \AttributeTok{{-}n} \StringTok{"}\VariableTok{$char}\StringTok{"}
\KeywordTok{\}}

\FunctionTok{generate\_random\_sequence()} \KeywordTok{\{}
    \BuiltInTok{local} \VariableTok{len}\OperatorTok{=}\VariableTok{$1}
    \BuiltInTok{local} \VariableTok{sequence}\OperatorTok{=}\StringTok{""}
    \ControlFlowTok{for} \KeywordTok{((}\VariableTok{i} \OperatorTok{=} \DecValTok{0}\KeywordTok{;} \VariableTok{i} \OperatorTok{\textless{}} \VariableTok{len}\KeywordTok{;} \VariableTok{i}\OperatorTok{++}\KeywordTok{));} \ControlFlowTok{do}
        \VariableTok{sequence}\OperatorTok{=}\StringTok{"}\VariableTok{$\{sequence\}$(}\ExtensionTok{generate\_random\_char}\VariableTok{)}\StringTok{"}
    \ControlFlowTok{done}
    \BuiltInTok{echo} \StringTok{"}\VariableTok{$sequence}\StringTok{"}
\KeywordTok{\}}

\VariableTok{result}\OperatorTok{=}\VariableTok{$(}\ExtensionTok{generate\_random\_sequence} \VariableTok{$LENGTH)}
\BuiltInTok{echo} \StringTok{"Случайная последовательность: }\VariableTok{$result}\StringTok{"}
\end{Highlighting}
\end{Shaded}
\end{frame}

\section{5. Результаты}\label{ux440ux435ux437ux443ux43bux44cux442ux430ux442ux44b}

\begin{frame}{5.1 Задание 1: Результаты работы семафора}
\phantomsection\label{ux437ux430ux434ux430ux43dux438ux435-1-ux440ux435ux437ux443ux43bux44cux442ux430ux442ux44b-ux440ux430ux431ux43eux442ux44b-ux441ux435ux43cux430ux444ux43eux440ux430}
\begin{itemize}[<+->]
\tightlist
\item
  Успешная реализация механизма синхронизации
\item
  Использование файла-блокировки для синхронизации
\item
  Возможность работы с несколькими процессами
\item
  Запуск в фоновом режиме с перенаправлением вывода
\end{itemize}
\end{frame}

\begin{frame}{5.2 Задание 2: Результаты работы man-команды}
\phantomsection\label{ux437ux430ux434ux430ux43dux438ux435-2-ux440ux435ux437ux443ux43bux44cux442ux430ux442ux44b-ux440ux430ux431ux43eux442ux44b-man-ux43aux43eux43cux430ux43dux434ux44b}
\begin{itemize}[<+->]
\tightlist
\item
  Успешная реализация команды man
\item
  Поиск справочных страниц в /usr/share/man/man1/
\item
  Обработка сжатых (.gz) файлов
\item
  Корректная обработка отсутствующих страниц
\end{itemize}
\end{frame}

\begin{frame}{5.3 Задание 3: Результаты генерации случайных
последовательностей}
\phantomsection\label{ux437ux430ux434ux430ux43dux438ux435-3-ux440ux435ux437ux443ux43bux44cux442ux430ux442ux44b-ux433ux435ux43dux435ux440ux430ux446ux438ux438-ux441ux43bux443ux447ux430ux439ux43dux44bux445-ux43fux43eux441ux43bux435ux434ux43eux432ux430ux442ux435ux43bux44cux43dux43eux441ux442ux435ux439}
\begin{itemize}[<+->]
\tightlist
\item
  Успешная генерация случайных последовательностей
\item
  Использование переменной \$RANDOM
\item
  Генерация букв латинского алфавита
\item
  Возможность задания длины последовательности
\end{itemize}
\end{frame}

\section{6. Выводы}\label{ux432ux44bux432ux43eux434ux44b}

\begin{frame}{6.1 Основные результаты}
\phantomsection\label{ux43eux441ux43dux43eux432ux43dux44bux435-ux440ux435ux437ux443ux43bux44cux442ux430ux442ux44b}
\begin{itemize}[<+->]
\tightlist
\item
  ✅ Реализован упрощённый механизм семафоров
\item
  ✅ Создан скрипт, реализующий функциональность команды man
\item
  ✅ Разработан генератор случайных последовательностей латинских букв
\end{itemize}
\end{frame}

\begin{frame}{6.2 Полученные навыки}
\phantomsection\label{ux43fux43eux43bux443ux447ux435ux43dux43dux44bux435-ux43dux430ux432ux44bux43aux438}
\begin{itemize}[<+->]
\tightlist
\item
  Работа с файловыми блокировками
\item
  Обработка системных справочных файлов
\item
  Генерация псевдослучайных данных
\item
  Создание продвинутых shell-скриптов
\end{itemize}
\end{frame}

\section{7. Спасибо за
внимание}\label{ux441ux43fux430ux441ux438ux431ux43e-ux437ux430-ux432ux43dux438ux43cux430ux43dux438ux435}




\end{document}
