% Options for packages loaded elsewhere
\PassOptionsToPackage{unicode}{hyperref}
\PassOptionsToPackage{hyphens}{url}
%
\documentclass[
  ignorenonframetext,
  aspectratio=169,
]{beamer}
\usepackage{pgfpages}
\setbeamertemplate{caption}[numbered]
\setbeamertemplate{caption label separator}{: }
\setbeamercolor{caption name}{fg=normal text.fg}
\beamertemplatenavigationsymbolshorizontal
% Prevent slide breaks in the middle of a paragraph
\widowpenalties 1 10000
\raggedbottom
\setbeamertemplate{part page}{
  \centering
  \begin{beamercolorbox}[sep=16pt,center]{part title}
    \usebeamerfont{part title}\insertpart\par
  \end{beamercolorbox}
}
\setbeamertemplate{section page}{
  \centering
  \begin{beamercolorbox}[sep=12pt,center]{part title}
    \usebeamerfont{section title}\insertsection\par
  \end{beamercolorbox}
}
\setbeamertemplate{subsection page}{
  \centering
  \begin{beamercolorbox}[sep=8pt,center]{part title}
    \usebeamerfont{subsection title}\insertsubsection\par
  \end{beamercolorbox}
}
\AtBeginPart{
  \frame{\partpage}
}
\AtBeginSection{
  \ifbibliography
  \else
    \frame{\sectionpage}
  \fi
}
\AtBeginSubsection{
  \frame{\subsectionpage}
}

\usepackage{amsmath,amssymb}
\usepackage{iftex}
\ifPDFTeX
  \usepackage[T1]{fontenc}
  \usepackage[utf8]{inputenc}
  \usepackage{textcomp} % provide euro and other symbols
\else % if luatex or xetex
  \usepackage{unicode-math}
  \defaultfontfeatures{Scale=MatchLowercase}
  \defaultfontfeatures[\rmfamily]{Ligatures=TeX,Scale=1}
\fi
\usepackage{lmodern}
\usetheme[]{Montpellier}
\usecolortheme{seagull}
\ifPDFTeX\else  
    % xetex/luatex font selection
\fi
% Use upquote if available, for straight quotes in verbatim environments
\IfFileExists{upquote.sty}{\usepackage{upquote}}{}
\IfFileExists{microtype.sty}{% use microtype if available
  \usepackage[]{microtype}
  \UseMicrotypeSet[protrusion]{basicmath} % disable protrusion for tt fonts
}{}
\usepackage{xcolor}
\newif\ifbibliography
\setlength{\emergencystretch}{3em} % prevent overfull lines
\setcounter{secnumdepth}{5}

\usepackage{color}
\usepackage{fancyvrb}
\newcommand{\VerbBar}{|}
\newcommand{\VERB}{\Verb[commandchars=\\\{\}]}
\DefineVerbatimEnvironment{Highlighting}{Verbatim}{commandchars=\\\{\}}
% Add ',fontsize=\small' for more characters per line
\usepackage{framed}
\definecolor{shadecolor}{RGB}{241,243,245}
\newenvironment{Shaded}{\begin{snugshade}}{\end{snugshade}}
\newcommand{\AlertTok}[1]{\textcolor[rgb]{0.68,0.00,0.00}{#1}}
\newcommand{\AnnotationTok}[1]{\textcolor[rgb]{0.37,0.37,0.37}{#1}}
\newcommand{\AttributeTok}[1]{\textcolor[rgb]{0.40,0.45,0.13}{#1}}
\newcommand{\BaseNTok}[1]{\textcolor[rgb]{0.68,0.00,0.00}{#1}}
\newcommand{\BuiltInTok}[1]{\textcolor[rgb]{0.00,0.23,0.31}{#1}}
\newcommand{\CharTok}[1]{\textcolor[rgb]{0.13,0.47,0.30}{#1}}
\newcommand{\CommentTok}[1]{\textcolor[rgb]{0.37,0.37,0.37}{#1}}
\newcommand{\CommentVarTok}[1]{\textcolor[rgb]{0.37,0.37,0.37}{\textit{#1}}}
\newcommand{\ConstantTok}[1]{\textcolor[rgb]{0.56,0.35,0.01}{#1}}
\newcommand{\ControlFlowTok}[1]{\textcolor[rgb]{0.00,0.23,0.31}{#1}}
\newcommand{\DataTypeTok}[1]{\textcolor[rgb]{0.68,0.00,0.00}{#1}}
\newcommand{\DecValTok}[1]{\textcolor[rgb]{0.68,0.00,0.00}{#1}}
\newcommand{\DocumentationTok}[1]{\textcolor[rgb]{0.37,0.37,0.37}{\textit{#1}}}
\newcommand{\ErrorTok}[1]{\textcolor[rgb]{0.68,0.00,0.00}{#1}}
\newcommand{\ExtensionTok}[1]{\textcolor[rgb]{0.00,0.23,0.31}{#1}}
\newcommand{\FloatTok}[1]{\textcolor[rgb]{0.68,0.00,0.00}{#1}}
\newcommand{\FunctionTok}[1]{\textcolor[rgb]{0.28,0.35,0.67}{#1}}
\newcommand{\ImportTok}[1]{\textcolor[rgb]{0.00,0.46,0.62}{#1}}
\newcommand{\InformationTok}[1]{\textcolor[rgb]{0.37,0.37,0.37}{#1}}
\newcommand{\KeywordTok}[1]{\textcolor[rgb]{0.00,0.23,0.31}{#1}}
\newcommand{\NormalTok}[1]{\textcolor[rgb]{0.00,0.23,0.31}{#1}}
\newcommand{\OperatorTok}[1]{\textcolor[rgb]{0.37,0.37,0.37}{#1}}
\newcommand{\OtherTok}[1]{\textcolor[rgb]{0.00,0.23,0.31}{#1}}
\newcommand{\PreprocessorTok}[1]{\textcolor[rgb]{0.68,0.00,0.00}{#1}}
\newcommand{\RegionMarkerTok}[1]{\textcolor[rgb]{0.00,0.23,0.31}{#1}}
\newcommand{\SpecialCharTok}[1]{\textcolor[rgb]{0.37,0.37,0.37}{#1}}
\newcommand{\SpecialStringTok}[1]{\textcolor[rgb]{0.13,0.47,0.30}{#1}}
\newcommand{\StringTok}[1]{\textcolor[rgb]{0.13,0.47,0.30}{#1}}
\newcommand{\VariableTok}[1]{\textcolor[rgb]{0.07,0.07,0.07}{#1}}
\newcommand{\VerbatimStringTok}[1]{\textcolor[rgb]{0.13,0.47,0.30}{#1}}
\newcommand{\WarningTok}[1]{\textcolor[rgb]{0.37,0.37,0.37}{\textit{#1}}}

\providecommand{\tightlist}{%
  \setlength{\itemsep}{0pt}\setlength{\parskip}{0pt}}\usepackage{longtable,booktabs,array}
\usepackage{calc} % for calculating minipage widths
\usepackage{caption}
% Make caption package work with longtable
\makeatletter
\def\fnum@table{\tablename~\thetable}
\makeatother
\usepackage{graphicx}
\makeatletter
\def\maxwidth{\ifdim\Gin@nat@width>\linewidth\linewidth\else\Gin@nat@width\fi}
\def\maxheight{\ifdim\Gin@nat@height>\textheight\textheight\else\Gin@nat@height\fi}
\makeatother
% Scale images if necessary, so that they will not overflow the page
% margins by default, and it is still possible to overwrite the defaults
% using explicit options in \includegraphics[width, height, ...]{}
\setkeys{Gin}{width=\maxwidth,height=\maxheight,keepaspectratio}
% Set default figure placement to htbp
\makeatletter
\def\fps@figure{htbp}
\makeatother
% definitions for citeproc citations
\NewDocumentCommand\citeproctext{}{}
\NewDocumentCommand\citeproc{mm}{%
  \begingroup\def\citeproctext{#2}\cite{#1}\endgroup}
\makeatletter
 % allow citations to break across lines
 \let\@cite@ofmt\@firstofone
 % avoid brackets around text for \cite:
 \def\@biblabel#1{}
 \def\@cite#1#2{{#1\if@tempswa , #2\fi}}
\makeatother
\newlength{\cslhangindent}
\setlength{\cslhangindent}{1.5em}
\newlength{\csllabelwidth}
\setlength{\csllabelwidth}{3em}
\newenvironment{CSLReferences}[2] % #1 hanging-indent, #2 entry-spacing
 {\begin{list}{}{%
  \setlength{\itemindent}{0pt}
  \setlength{\leftmargin}{0pt}
  \setlength{\parsep}{0pt}
  % turn on hanging indent if param 1 is 1
  \ifodd #1
   \setlength{\leftmargin}{\cslhangindent}
   \setlength{\itemindent}{-1\cslhangindent}
  \fi
  % set entry spacing
  \setlength{\itemsep}{#2\baselineskip}}}
 {\end{list}}
\usepackage{calc}
\newcommand{\CSLBlock}[1]{\hfill\break\parbox[t]{\linewidth}{\strut\ignorespaces#1\strut}}
\newcommand{\CSLLeftMargin}[1]{\parbox[t]{\csllabelwidth}{\strut#1\strut}}
\newcommand{\CSLRightInline}[1]{\parbox[t]{\linewidth - \csllabelwidth}{\strut#1\strut}}
\newcommand{\CSLIndent}[1]{\hspace{\cslhangindent}#1}

\usepackage{libertine}
\makeatletter
\@ifpackageloaded{caption}{}{\usepackage{caption}}
\AtBeginDocument{%
\ifdefined\contentsname
  \renewcommand*\contentsname{Содержание}
\else
  \newcommand\contentsname{Содержание}
\fi
\ifdefined\listfigurename
  \renewcommand*\listfigurename{Список иллюстраций}
\else
  \newcommand\listfigurename{Список иллюстраций}
\fi
\ifdefined\listtablename
  \renewcommand*\listtablename{Список таблиц}
\else
  \newcommand\listtablename{Список таблиц}
\fi
\ifdefined\figurename
  \renewcommand*\figurename{Рисунок}
\else
  \newcommand\figurename{Рисунок}
\fi
\ifdefined\tablename
  \renewcommand*\tablename{Таблица}
\else
  \newcommand\tablename{Таблица}
\fi
}
\@ifpackageloaded{float}{}{\usepackage{float}}
\floatstyle{ruled}
\@ifundefined{c@chapter}{\newfloat{codelisting}{h}{lop}}{\newfloat{codelisting}{h}{lop}[chapter]}
\floatname{codelisting}{Список}
\newcommand*\listoflistings{\listof{codelisting}{Листинги}}
\makeatother
\makeatletter
\makeatother
\makeatletter
\@ifpackageloaded{caption}{}{\usepackage{caption}}
\@ifpackageloaded{subcaption}{}{\usepackage{subcaption}}
\makeatother
\ifLuaTeX
\usepackage[bidi=basic]{babel}
\else
\usepackage[bidi=default]{babel}
\fi
\babelprovide[main,import]{russian}
\babelprovide[import]{english}
% get rid of language-specific shorthands (see #6817):
\let\LanguageShortHands\languageshorthands
\def\languageshorthands#1{}
\ifLuaTeX
  \usepackage{selnolig}  % disable illegal ligatures
\fi
\usepackage{csquotes}
\usepackage{bookmark}

\IfFileExists{xurl.sty}{\usepackage{xurl}}{} % add URL line breaks if available
\urlstyle{same} % disable monospaced font for URLs
\hypersetup{
  pdftitle={Лабораторная работа №3},
  pdfauthor={Мохамед Муса},
  pdflang={ru-RU},
  hidelinks,
  pdfcreator={LaTeX via pandoc}}

\title{Лабораторная работа №3}
\subtitle{Оформление отчетов в Markdown}
\author{Мохамед Муса}
\date{2025-10-08}

\begin{document}
\frame{\titlepage}

\renewcommand*\contentsname{Содержание}
\begin{frame}[allowframebreaks]
  \frametitle{Содержание}
  \tableofcontents[hideallsubsections]
\end{frame}
\begin{frame}{1. Цель работы}
\phantomsection\label{ux446ux435ux43bux44c-ux440ux430ux431ux43eux442ux44b}
Освоить процедуру оформления отчетов с помощью легковесного языка
разметки Markdown и системы сборки Make.
\end{frame}

\begin{frame}{2. Задачи}
\phantomsection\label{ux437ux430ux434ux430ux447ux438}
\begin{columns}[c]
\begin{column}{0.6\textwidth}
\begin{enumerate}[<+->]
\tightlist
\item
  Изучить синтаксис языка разметки Markdown
\item
  Освоить систему автоматизации сборки Make
\item
  Научиться создавать отчеты с использованием Markdown
\item
  Практически применить знания для генерации документов
\end{enumerate}
\end{column}

\begin{column}{0.4\textwidth}
\end{column}
\end{columns}
\end{frame}

\begin{frame}{3. Теоретическое введение}
\phantomsection\label{ux442ux435ux43eux440ux435ux442ux438ux447ux435ux441ux43aux43eux435-ux432ux432ux435ux434ux435ux43dux438ux435}
\begin{block}{3.1 Язык разметки Markdown}
\phantomsection\label{ux44fux437ux44bux43a-ux440ux430ux437ux43cux435ux442ux43aux438-markdown}
\begin{columns}[c]
\begin{column}{0.6\textwidth}
\textbf{Markdown} --- легковесный язык разметки для форматирования
текста

\textbf{Преимущества:} - Простой синтаксис - Читаемость в исходном виде
- Независимость от платформы - Широкая поддержка
\end{column}

\begin{column}{0.4\textwidth}
\end{column}
\end{columns}
\end{block}

\begin{block}{3.2 Система сборки Make}
\phantomsection\label{ux441ux438ux441ux442ux435ux43cux430-ux441ux431ux43eux440ux43aux438-make}
\begin{columns}[c]
\begin{column}{0.6\textwidth}
\textbf{Make} --- система автоматизации для управления зависимостями
между файлами

\textbf{Возможности:} - Автоматизация компиляции - Управление
зависимостями - Упрощение повторяющихся задач - Интеграция с различными
инструментами
\end{column}

\begin{column}{0.4\textwidth}
\end{column}
\end{columns}
\end{block}
\end{frame}

\begin{frame}[fragile]{4. Синтаксис Markdown}
\phantomsection\label{ux441ux438ux43dux442ux430ux43aux441ux438ux441-markdown}
\begin{block}{4.1 Основные элементы}
\phantomsection\label{ux43eux441ux43dux43eux432ux43dux44bux435-ux44dux43bux435ux43cux435ux43dux442ux44b}
\begin{columns}[c]
\begin{column}{0.5\textwidth}
\textbf{Заголовки:}

\begin{Shaded}
\begin{Highlighting}[]
\FunctionTok{\# Заголовок 1}
\FunctionTok{\#\# Заголовок 2}
\FunctionTok{\#\#\# Заголовок 3}
\end{Highlighting}
\end{Shaded}

\textbf{Форматирование:}

\begin{Shaded}
\begin{Highlighting}[]
\NormalTok{**полужирный текст**}
\NormalTok{*курсивный текст*}
\InformationTok{\textasciigrave{}код в строке\textasciigrave{}}
\end{Highlighting}
\end{Shaded}
\end{column}

\begin{column}{0.5\textwidth}
\textbf{Списки:}

\begin{Shaded}
\begin{Highlighting}[]
\SpecialStringTok{{-} }\NormalTok{Элемент 1}
\SpecialStringTok{{-} }\NormalTok{Элемент 2}
\SpecialStringTok{  {-} }\NormalTok{Подэлемент}

\SpecialStringTok{1. }\NormalTok{Нумерованный}
\SpecialStringTok{2. }\NormalTok{Список}
\end{Highlighting}
\end{Shaded}

\textbf{Ссылки:}

\begin{Shaded}
\begin{Highlighting}[]
\CommentTok{[}\OtherTok{текст ссылки}\CommentTok{](URL)}
\AlertTok{![альтернативный текст](путь к изображению)}
\end{Highlighting}
\end{Shaded}
\end{column}
\end{columns}
\end{block}

\begin{block}{4.2 Таблицы в Markdown}
\phantomsection\label{ux442ux430ux431ux43bux438ux446ux44b-ux432-markdown}
\begin{Shaded}
\begin{Highlighting}[]
\NormalTok{| Заголовок 1 | Заголовок 2 |}
\NormalTok{|{-}{-}{-}{-}{-}{-}{-}{-}{-}{-}{-}{-}{-}|{-}{-}{-}{-}{-}{-}{-}{-}{-}{-}{-}{-}{-}|}
\NormalTok{| Ячейка 1    | Ячейка 2    |}
\NormalTok{| Ячейка 3    | Ячейка 4    |}
\end{Highlighting}
\end{Shaded}

Результат:

\begin{longtable}[]{@{}ll@{}}
\toprule\noalign{}
Заголовок 1 & Заголовок 2 \\
\midrule\noalign{}
\endhead
Ячейка 1 & Ячейка 2 \\
Ячейка 3 & Ячейка 4 \\
\bottomrule\noalign{}
\end{longtable}
\end{block}
\end{frame}

\begin{frame}[fragile]{5. Выполнение работы}
\phantomsection\label{ux432ux44bux43fux43eux43bux43dux435ux43dux438ux435-ux440ux430ux431ux43eux442ux44b}
\begin{block}{5.1 Создание Markdown документа}
\phantomsection\label{ux441ux43eux437ux434ux430ux43dux438ux435-markdown-ux434ux43eux43aux443ux43cux435ux43dux442ux430}
\begin{columns}[c]
\begin{column}{0.6\textwidth}
\textbf{Структура отчета в Markdown:}

\begin{Shaded}
\begin{Highlighting}[]
\PreprocessorTok{{-}{-}{-}}
\CommentTok{\#\# Author}
\FunctionTok{author}\KeywordTok{:}
\AttributeTok{  }\FunctionTok{name}\KeywordTok{:}\AttributeTok{ Мохамед Ахмед Муса}
\AttributeTok{  }\FunctionTok{email}\KeywordTok{:}\AttributeTok{ }\StringTok{"1032248286@pfur.ru"}

\CommentTok{\#\# Title}
\FunctionTok{title}\KeywordTok{:}\AttributeTok{ }\StringTok{"Лабораторная работа №3"}
\PreprocessorTok{{-}{-}{-}}

\CommentTok{\# Введение}

\AttributeTok{Текст в Markdown формате.}
\end{Highlighting}
\end{Shaded}
\end{column}

\begin{column}{0.4\textwidth}
\end{column}
\end{columns}
\end{block}

\begin{block}{5.2 Система сборки Make}
\phantomsection\label{ux441ux438ux441ux442ux435ux43cux430-ux441ux431ux43eux440ux43aux438-make-1}
\begin{columns}[c]
\begin{column}{0.6\textwidth}
\textbf{Makefile для автоматизации:}

\begin{Shaded}
\begin{Highlighting}[]
\DecValTok{all:}
\ErrorTok{    }\NormalTok{quarto render}

\DecValTok{clean:}
\ErrorTok{    }\NormalTok{rm {-}rf \_output}

\DecValTok{install:}
\ErrorTok{    }\NormalTok{sudo apt install quarto}
\end{Highlighting}
\end{Shaded}

\textbf{Команды:} - \texttt{make} - сборка документов -
\texttt{make\ clean} - очистка
\end{column}

\begin{column}{0.4\textwidth}
\end{column}
\end{columns}
\end{block}

\begin{block}{5.3 Генерация документов}
\phantomsection\label{ux433ux435ux43dux435ux440ux430ux446ux438ux44f-ux434ux43eux43aux443ux43cux435ux43dux442ux43eux432}
\begin{columns}[c]
\begin{column}{0.6\textwidth}
\textbf{Форматы вывода:} - PDF документы - HTML презентации - DOCX файлы
- Reveal.js слайды

\textbf{Преимущества:} - Единый источник - Множественные форматы -
Автоматическая сборка
\end{column}

\begin{column}{0.4\textwidth}
\end{column}
\end{columns}
\end{block}
\end{frame}

\begin{frame}{6. Выводы}
\phantomsection\label{ux432ux44bux432ux43eux434ux44b}
\begin{block}{6.1 Достигнутые результаты}
\phantomsection\label{ux434ux43eux441ux442ux438ux433ux43dux443ux442ux44bux435-ux440ux435ux437ux443ux43bux44cux442ux430ux442ux44b}
\begin{columns}[c]
\begin{column}{0.6\textwidth}
✅ \textbf{Изучен синтаксис Markdown} - Заголовки, списки, таблицы -
Форматирование текста - Ссылки и изображения

✅ \textbf{Освоена система Make} - Автоматизация сборки - Управление
зависимостями - Интеграция с Quarto

✅ \textbf{Создан структурированный отчет} - Академический формат -
Множественные форматы вывода
\end{column}

\begin{column}{0.4\textwidth}
\end{column}
\end{columns}
\end{block}

\begin{block}{6.2 Полученные навыки}
\phantomsection\label{ux43fux43eux43bux443ux447ux435ux43dux43dux44bux435-ux43dux430ux432ux44bux43aux438}
\begin{columns}[c]
\begin{column}{0.6\textwidth}
\textbf{Технические навыки:} - Работа с Markdown - Автоматизация с Make
- Генерация документов

\textbf{Практические навыки:} - Структурирование документов -
Академическое письмо - Рабочий процесс разработки
\end{column}

\begin{column}{0.4\textwidth}
\end{column}
\end{columns}
\end{block}
\end{frame}

\begin{frame}{7. Список литературы}
\phantomsection\label{ux441ux43fux438ux441ux43eux43a-ux43bux438ux442ux435ux440ux430ux442ux443ux440ux44b}
\phantomsection\label{refs}
\begin{CSLReferences}{0}{1}
\begin{itemize}[<+->]
\tightlist
\item
  Markdown Guide: \url{https://www.markdownguide.org/}
\item
  Make Manual: \url{https://www.gnu.org/software/make/manual/}
\item
  Quarto Documentation: \url{https://quarto.org/}
\item
  John Gruber: \url{https://daringfireball.net/projects/markdown/syntax}
\end{itemize}

\end{CSLReferences}
\end{frame}



\end{document}
