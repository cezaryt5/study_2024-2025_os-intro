% Options for packages loaded elsewhere
\PassOptionsToPackage{unicode}{hyperref}
\PassOptionsToPackage{hyphens}{url}
%
\documentclass[
  12pt,
  a4paper,
  DIV=11,
  numbers=noendperiod]{scrreprt}

\usepackage{amsmath,amssymb}
\usepackage{setspace}
\usepackage{iftex}
\ifPDFTeX
  \usepackage[T1]{fontenc}
  \usepackage[utf8]{inputenc}
  \usepackage{textcomp} % provide euro and other symbols
\else % if luatex or xetex
  \usepackage{unicode-math}
  \defaultfontfeatures{Scale=MatchLowercase}
  \defaultfontfeatures[\rmfamily]{Ligatures=TeX,Scale=1}
\fi
\usepackage{lmodern}
\ifPDFTeX\else  
    % xetex/luatex font selection
\fi
% Use upquote if available, for straight quotes in verbatim environments
\IfFileExists{upquote.sty}{\usepackage{upquote}}{}
\IfFileExists{microtype.sty}{% use microtype if available
  \usepackage[]{microtype}
  \UseMicrotypeSet[protrusion]{basicmath} % disable protrusion for tt fonts
}{}
\usepackage{xcolor}
\setlength{\emergencystretch}{3em} % prevent overfull lines
\setcounter{secnumdepth}{5}
% Make \paragraph and \subparagraph free-standing
\ifx\paragraph\undefined\else
  \let\oldparagraph\paragraph
  \renewcommand{\paragraph}[1]{\oldparagraph{#1}\mbox{}}
\fi
\ifx\subparagraph\undefined\else
  \let\oldsubparagraph\subparagraph
  \renewcommand{\subparagraph}[1]{\oldsubparagraph{#1}\mbox{}}
\fi

\usepackage{color}
\usepackage{fancyvrb}
\newcommand{\VerbBar}{|}
\newcommand{\VERB}{\Verb[commandchars=\\\{\}]}
\DefineVerbatimEnvironment{Highlighting}{Verbatim}{commandchars=\\\{\}}
% Add ',fontsize=\small' for more characters per line
\usepackage{framed}
\definecolor{shadecolor}{RGB}{241,243,245}
\newenvironment{Shaded}{\begin{snugshade}}{\end{snugshade}}
\newcommand{\AlertTok}[1]{\textcolor[rgb]{0.68,0.00,0.00}{#1}}
\newcommand{\AnnotationTok}[1]{\textcolor[rgb]{0.37,0.37,0.37}{#1}}
\newcommand{\AttributeTok}[1]{\textcolor[rgb]{0.40,0.45,0.13}{#1}}
\newcommand{\BaseNTok}[1]{\textcolor[rgb]{0.68,0.00,0.00}{#1}}
\newcommand{\BuiltInTok}[1]{\textcolor[rgb]{0.00,0.23,0.31}{#1}}
\newcommand{\CharTok}[1]{\textcolor[rgb]{0.13,0.47,0.30}{#1}}
\newcommand{\CommentTok}[1]{\textcolor[rgb]{0.37,0.37,0.37}{#1}}
\newcommand{\CommentVarTok}[1]{\textcolor[rgb]{0.37,0.37,0.37}{\textit{#1}}}
\newcommand{\ConstantTok}[1]{\textcolor[rgb]{0.56,0.35,0.01}{#1}}
\newcommand{\ControlFlowTok}[1]{\textcolor[rgb]{0.00,0.23,0.31}{#1}}
\newcommand{\DataTypeTok}[1]{\textcolor[rgb]{0.68,0.00,0.00}{#1}}
\newcommand{\DecValTok}[1]{\textcolor[rgb]{0.68,0.00,0.00}{#1}}
\newcommand{\DocumentationTok}[1]{\textcolor[rgb]{0.37,0.37,0.37}{\textit{#1}}}
\newcommand{\ErrorTok}[1]{\textcolor[rgb]{0.68,0.00,0.00}{#1}}
\newcommand{\ExtensionTok}[1]{\textcolor[rgb]{0.00,0.23,0.31}{#1}}
\newcommand{\FloatTok}[1]{\textcolor[rgb]{0.68,0.00,0.00}{#1}}
\newcommand{\FunctionTok}[1]{\textcolor[rgb]{0.28,0.35,0.67}{#1}}
\newcommand{\ImportTok}[1]{\textcolor[rgb]{0.00,0.46,0.62}{#1}}
\newcommand{\InformationTok}[1]{\textcolor[rgb]{0.37,0.37,0.37}{#1}}
\newcommand{\KeywordTok}[1]{\textcolor[rgb]{0.00,0.23,0.31}{#1}}
\newcommand{\NormalTok}[1]{\textcolor[rgb]{0.00,0.23,0.31}{#1}}
\newcommand{\OperatorTok}[1]{\textcolor[rgb]{0.37,0.37,0.37}{#1}}
\newcommand{\OtherTok}[1]{\textcolor[rgb]{0.00,0.23,0.31}{#1}}
\newcommand{\PreprocessorTok}[1]{\textcolor[rgb]{0.68,0.00,0.00}{#1}}
\newcommand{\RegionMarkerTok}[1]{\textcolor[rgb]{0.00,0.23,0.31}{#1}}
\newcommand{\SpecialCharTok}[1]{\textcolor[rgb]{0.37,0.37,0.37}{#1}}
\newcommand{\SpecialStringTok}[1]{\textcolor[rgb]{0.13,0.47,0.30}{#1}}
\newcommand{\StringTok}[1]{\textcolor[rgb]{0.13,0.47,0.30}{#1}}
\newcommand{\VariableTok}[1]{\textcolor[rgb]{0.07,0.07,0.07}{#1}}
\newcommand{\VerbatimStringTok}[1]{\textcolor[rgb]{0.13,0.47,0.30}{#1}}
\newcommand{\WarningTok}[1]{\textcolor[rgb]{0.37,0.37,0.37}{\textit{#1}}}

\providecommand{\tightlist}{%
  \setlength{\itemsep}{0pt}\setlength{\parskip}{0pt}}\usepackage{longtable,booktabs,array}
\usepackage{calc} % for calculating minipage widths
% Correct order of tables after \paragraph or \subparagraph
\usepackage{etoolbox}
\makeatletter
\patchcmd\longtable{\par}{\if@noskipsec\mbox{}\fi\par}{}{}
\makeatother
% Allow footnotes in longtable head/foot
\IfFileExists{footnotehyper.sty}{\usepackage{footnotehyper}}{\usepackage{footnote}}
\makesavenoteenv{longtable}
\usepackage{graphicx}
\makeatletter
\def\maxwidth{\ifdim\Gin@nat@width>\linewidth\linewidth\else\Gin@nat@width\fi}
\def\maxheight{\ifdim\Gin@nat@height>\textheight\textheight\else\Gin@nat@height\fi}
\makeatother
% Scale images if necessary, so that they will not overflow the page
% margins by default, and it is still possible to overwrite the defaults
% using explicit options in \includegraphics[width, height, ...]{}
\setkeys{Gin}{width=\maxwidth,height=\maxheight,keepaspectratio}
% Set default figure placement to htbp
\makeatletter
\def\fps@figure{htbp}
\makeatother

\KOMAoption{captions}{tableheading}
\usepackage{indentfirst}
\usepackage{float}
\floatplacement{figure}{H}
\usepackage{libertine}
\makeatletter
\@ifpackageloaded{caption}{}{\usepackage{caption}}
\AtBeginDocument{%
\ifdefined\contentsname
  \renewcommand*\contentsname{Содержание}
\else
  \newcommand\contentsname{Содержание}
\fi
\ifdefined\listfigurename
  \renewcommand*\listfigurename{Список иллюстраций}
\else
  \newcommand\listfigurename{Список иллюстраций}
\fi
\ifdefined\listtablename
  \renewcommand*\listtablename{Список таблиц}
\else
  \newcommand\listtablename{Список таблиц}
\fi
\ifdefined\figurename
  \renewcommand*\figurename{Рисунок}
\else
  \newcommand\figurename{Рисунок}
\fi
\ifdefined\tablename
  \renewcommand*\tablename{Таблица}
\else
  \newcommand\tablename{Таблица}
\fi
}
\@ifpackageloaded{float}{}{\usepackage{float}}
\floatstyle{ruled}
\@ifundefined{c@chapter}{\newfloat{codelisting}{h}{lop}}{\newfloat{codelisting}{h}{lop}[chapter]}
\floatname{codelisting}{Список}
\newcommand*\listoflistings{\listof{codelisting}{Листинги}}
\makeatother
\makeatletter
\makeatother
\makeatletter
\@ifpackageloaded{caption}{}{\usepackage{caption}}
\@ifpackageloaded{subcaption}{}{\usepackage{subcaption}}
\makeatother
\ifLuaTeX
\usepackage[bidi=basic]{babel}
\else
\usepackage[bidi=default]{babel}
\fi
\babelprovide[main,import]{russian}
\babelprovide[import]{english}
% get rid of language-specific shorthands (see #6817):
\let\LanguageShortHands\languageshorthands
\def\languageshorthands#1{}
\ifLuaTeX
  \usepackage{selnolig}  % disable illegal ligatures
\fi
\usepackage[backend=biber,langhook=extras,autolang=other*]{biblatex}
\addbibresource{bib/cite.bib}
\usepackage{csquotes}
\usepackage{bookmark}

\IfFileExists{xurl.sty}{\usepackage{xurl}}{} % add URL line breaks if available
\urlstyle{same} % disable monospaced font for URLs
\hypersetup{
  pdftitle={Лабораторная работа №3},
  pdfauthor={Мохамед Муса},
  pdflang={ru-RU},
  hidelinks,
  pdfcreator={LaTeX via pandoc}}

\title{Лабораторная работа №3}
\usepackage{etoolbox}
\makeatletter
\providecommand{\subtitle}[1]{% add subtitle to \maketitle
  \apptocmd{\@title}{\par {\large #1 \par}}{}{}
}
\makeatother
\subtitle{Markdown}
\author{Мохамед Муса}
\date{2025-10-08}

\begin{document}
\maketitle

\renewcommand*\contentsname{Содержание}
{
\setcounter{tocdepth}{1}
\tableofcontents
}
\listoffigures
\listoftables
\setstretch{1.5}
\chapter{Введение}\label{ux432ux432ux435ux434ux435ux43dux438ux435}

\section{Цель
работы}\label{ux446ux435ux43bux44c-ux440ux430ux431ux43eux442ux44b}

Освоить процедуру оформления отчетов с помощью легковесного языка
разметки Markdown и системы сборки Make, научиться создавать
структурированные документы с автоматической генерацией выходных
форматов.

\section{Задачи}\label{ux437ux430ux434ux430ux447ux438}

\begin{enumerate}
\def\labelenumi{\arabic{enumi}.}
\tightlist
\item
  Изучить синтаксис языка разметки Markdown
\item
  Освоить систему автоматизации сборки Make
\item
  Научиться создавать отчеты с использованием Markdown
\item
  Практически применить полученные знания для генерации документов в
  различных форматах
\end{enumerate}

\chapter{Теоретическое
введение}\label{ux442ux435ux43eux440ux435ux442ux438ux447ux435ux441ux43aux43eux435-ux432ux432ux435ux434ux435ux43dux438ux435}

\section{Язык разметки
Markdown}\label{ux44fux437ux44bux43a-ux440ux430ux437ux43cux435ux442ux43aux438-markdown}

Markdown --- это легковесный язык разметки, созданный для форматирования
текста с помощью простого синтаксиса. Он позволяет создавать
структурированные документы, которые легко читаются как в исходном виде,
так и после преобразования в другие форматы.

\section{Основные элементы
Markdown}\label{ux43eux441ux43dux43eux432ux43dux44bux435-ux44dux43bux435ux43cux435ux43dux442ux44b-markdown}

В таблице Таблица~\ref{tbl-markdown} приведены основные элементы
синтаксиса Markdown.

\begin{longtable}[]{@{}lll@{}}
\caption{Основные элементы синтаксиса
Markdown}\label{tbl-markdown}\tabularnewline
\toprule\noalign{}
Элемент & Синтаксис & Результат \\
\midrule\noalign{}
\endfirsthead
\toprule\noalign{}
Элемент & Синтаксис & Результат \\
\midrule\noalign{}
\endhead
\bottomrule\noalign{}
\endlastfoot
Заголовок 1 & \texttt{\#\ Заголовок} & \# Заголовок \\
Заголовок 2 & \texttt{\#\#\ Заголовок} & \#\# Заголовок \\
Полужирный & \texttt{**текст**} & \textbf{текст} \\
Курсив & \texttt{*текст*} & \emph{текст} \\
Код & \texttt{\textasciigrave{}код\textasciigrave{}} & \texttt{код} \\
Ссылка & \texttt{{[}текст{]}(URL)} & \href{URL}{текст} \\
Изображение & \texttt{!{[}alt{]}(путь)} & \includegraphics{путь.pdf} \\
Список & \texttt{-\ элемент} & - элемент \\
\end{longtable}

\section{Система сборки
Make}\label{ux441ux438ux441ux442ux435ux43cux430-ux441ux431ux43eux440ux43aux438-make}

Make --- это система автоматизации сборки, которая позволяет описывать
зависимости между файлами и командами для их обработки. Она широко
используется для автоматизации компиляции программ, генерации
документации и других повторяющихся задач.

\section{Quarto для научных
документов}\label{quarto-ux434ux43bux44f-ux43dux430ux443ux447ux43dux44bux445-ux434ux43eux43aux443ux43cux435ux43dux442ux43eux432}

Quarto --- это система публикации научных и технических документов,
которая сочетает Markdown с возможностями генерации различных выходных
форматов (PDF, HTML, DOCX и др.).

\chapter{Выполнение лабораторной
работы}\label{ux432ux44bux43fux43eux43bux43dux435ux43dux438ux435-ux43bux430ux431ux43eux440ux430ux442ux43eux440ux43dux43eux439-ux440ux430ux431ux43eux442ux44b}

\section{Создание Markdown
документа}\label{ux441ux43eux437ux434ux430ux43dux438ux435-markdown-ux434ux43eux43aux443ux43cux435ux43dux442ux430}

Был создан отчет по лабораторной работе с использованием языка разметки
Markdown. Документ структурирован в соответствии с академическими
стандартами.

\subsection{Структура отчета в
Markdown}\label{ux441ux442ux440ux443ux43aux442ux443ux440ux430-ux43eux442ux447ux435ux442ux430-ux432-markdown}

\begin{Shaded}
\begin{Highlighting}[]
\CommentTok{{-}{-}{-}}
\CommentTok{\#\# Author}
\AnnotationTok{author:}
\CommentTok{  name: Мохамед Ахмед Муса}
\CommentTok{  degrees: Студент}
\CommentTok{  student\_number: "1032248286"}
\CommentTok{  group: "НКАбд{-}05{-}23"}
\CommentTok{  email: "1032248286@pfur.ru"}

\CommentTok{\#\# Title}
\AnnotationTok{title:}\CommentTok{ "Лабораторная работа №3"}
\AnnotationTok{subtitle:}\CommentTok{ "Оформление отчетов в Markdown"}
\CommentTok{{-}{-}{-}}

\FunctionTok{\# Введение}

\FunctionTok{\#\# Цель работы}

\NormalTok{Освоить процедуру оформления отчетов с помощью Markdown.}

\FunctionTok{\# Теоретическое введение}

\FunctionTok{\#\# Markdown}

\NormalTok{Markdown {-}{-}{-} легковесный язык разметки.}

\FunctionTok{\# Выполнение работы}

\NormalTok{Текст лабораторной работы.}

\FunctionTok{\# Выводы}

\NormalTok{Итоги работы.}
\end{Highlighting}
\end{Shaded}

\section{Работа с системой
Make}\label{ux440ux430ux431ux43eux442ux430-ux441-ux441ux438ux441ux442ux435ux43cux43eux439-make}

Была изучена система автоматизации сборки Make для генерации документов
в различных форматах.

\subsection{Makefile для
отчетов}\label{makefile-ux434ux43bux44f-ux43eux442ux447ux435ux442ux43eux432}

\begin{Shaded}
\begin{Highlighting}[]
\DecValTok{all:}
\ErrorTok{    }\NormalTok{quarto render}

\DecValTok{clean:}
\ErrorTok{    }\NormalTok{rm {-}rf \_output}

\DecValTok{install:}
\ErrorTok{    }\NormalTok{sudo apt install quarto}
\end{Highlighting}
\end{Shaded}

\section{Генерация выходных
форматов}\label{ux433ux435ux43dux435ux440ux430ux446ux438ux44f-ux432ux44bux445ux43eux434ux43dux44bux445-ux444ux43eux440ux43cux430ux442ux43eux432}

Выполнена генерация отчета в различные форматы с помощью Quarto:

\begin{Shaded}
\begin{Highlighting}[]
\CommentTok{\# Установка Quarto}
\FunctionTok{sudo}\NormalTok{ apt install quarto}

\CommentTok{\# Инициализация проекта}
\ExtensionTok{quarto}\NormalTok{ create project}

\CommentTok{\# Сборка документа}
\FunctionTok{make}\NormalTok{ all}

\CommentTok{\# Очистка временных файлов}
\FunctionTok{make}\NormalTok{ clean}
\end{Highlighting}
\end{Shaded}

\section{Практическая
работа}\label{ux43fux440ux430ux43aux442ux438ux447ux435ux441ux43aux430ux44f-ux440ux430ux431ux43eux442ux430}

Выполнены практические упражнения по созданию документов:

\begin{enumerate}
\def\labelenumi{\arabic{enumi}.}
\tightlist
\item
  \textbf{Создание Markdown документа} с различными элементами
  форматирования
\item
  \textbf{Настройка Makefile} для автоматизации сборки
\item
  \textbf{Генерация PDF} из Markdown с помощью Quarto
\item
  \textbf{Генерация HTML} презентации
\item
  \textbf{Работа с изображениями} и ссылками в документе
\end{enumerate}

\chapter{Заключение}\label{ux437ux430ux43aux43bux44eux447ux435ux43dux438ux435}

\section{Оценка выполнения
задач}\label{ux43eux446ux435ux43dux43aux430-ux432ux44bux43fux43eux43bux43dux435ux43dux438ux44f-ux437ux430ux434ux430ux447}

В ходе лабораторной работы были успешно выполнены все поставленные
задачи:

✅ \textbf{Изучен синтаксис Markdown} - освоены основные элементы
форматирования текста, заголовки, списки, таблицы, ссылки и изображения

✅ \textbf{Освоена система Make} - получен практический опыт создания
Makefile для автоматизации сборки документов

✅ \textbf{Научены создавать структурированные отчеты} - создан
полноценный академический отчет с использованием Markdown разметки

✅ \textbf{Применены знания на практике} - выполнена генерация
документов в различные форматы с помощью Quarto

\section{Полученные
навыки}\label{ux43fux43eux43bux443ux447ux435ux43dux43dux44bux435-ux43dux430ux432ux44bux43aux438}

В результате выполнения лабораторной работы получены фундаментальные
навыки работы с современными инструментами документирования:

\begin{itemize}
\tightlist
\item
  \textbf{Markdown} для быстрого и удобного форматирования текста
\item
  \textbf{Make} для автоматизации процесса сборки документов
\item
  \textbf{Quarto} для генерации документов в различные форматы
\item
  \textbf{Git} интеграция для управления версиями документации
\end{itemize}

\section{Рекомендации}\label{ux440ux435ux43aux43eux43cux435ux43dux434ux430ux446ux438ux438}

Для закрепления полученных навыков рекомендуется: - Создавать все
последующие отчеты используя Markdown и Make - Изучить дополнительные
возможности Quarto (презентации, сайты, книги) - Освоить продвинутые
возможности Markdown (диаграммы, формулы) - Интегрировать систему сборки
в рабочий процесс разработки

\chapter*{Список
литературы}\label{ux441ux43fux438ux441ux43eux43a-ux43bux438ux442ux435ux440ux430ux442ux443ux440ux44b}
\addcontentsline{toc}{chapter}{Список литературы}

\printbibliography[heading=none]




\end{document}
